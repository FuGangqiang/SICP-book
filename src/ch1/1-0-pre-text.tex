% 1-0-pre-text.tex

\fixme{quotation}

\startnarrower[4*left]
The acts of the mind,
wherein it exerts its power over simple ideas,
are chiefly these three:
1. Combining several simple ideas into one compound one, and thus all complex ideas are made.
2. The second is bringing two ideas, whether simple or complex, together,
and setting them by one another so as to take a view of them at once,
without uniting them into one, by which it gets all its ideas of relations.
3. The third is separating them from all other ideas that accompany them in their real existence:
this is called abstraction, and thus all its general ideas are made.

John Locke, An Essay Concerning Human Understanding (1690)
\stopnarrower

We are about to study the idea of a computational process.
Computational processes are abstract beings that inhabit computers.
As they evolve, processes manipulate other abstract things called data.
The evolution of a process is directed by a pattern of rules called a program.
People create programs to direct processes.
In effect, we conjure the spirits of the computer with our spells.

A computational process is indeed much like a sorcerer's idea of a spirit.
It cannot be seen or touched. It is not composed of matter at all.
However, it is very real.
It can perform intellectual work.
It can answer questions.
It can affect the world by disbursing money at a bank or by controlling a robot arm in a factory.
The programs we use to conjure processes are like a sorcerer's spells.
They are carefully composed from symbolic expressions in arcane and esoteric programming languages
that prescribe the tasks we want our processes to perform.

A computational process, in a correctly working computer, executes programs precisely and accurately.
Thus, like the sorcerer's apprentice,
novice programmers must learn to understand and to anticipate the consequences of their conjuring.
Even small errors (usually called bugs or glitches) in programs can have complex and unanticipated consequences.

Fortunately,
learning to program is considerably less dangerous than learning sorcery,
because the spirits we deal with are conveniently contained in a secure way.
Real-world programming, however, requires care, expertise, and wisdom.
A small bug in a computer-aided design program, for example,
can lead to the catastrophic collapse of an airplane or a dam or the self-destruction of an industrial robot.

Master software engineers have the ability to organize programs
so that they can be reasonably sure that the resulting processes will perform the tasks intended.
They can visualize the behavior of their systems in advance.
They know how to structure programs so that unanticipated problems do not lead to catastrophic consequences,
and when problems do arise, they can debug their programs.
Well-designed computational systems, like well-designed automobiles or nuclear reactors,
are designed in a modular manner, so that the parts can be constructed, replaced, and debugged separately.

\subject{Programming in Lisp}
\marking[section]{Programming in Lisp}

We need an appropriate language for describing processes,
and we will use for this purpose the programming language Lisp.
Just as our everyday thoughts are usually expressed in our natural language (such as English, French, or Japanese),
and descriptions of quantitative phenomena are expressed with mathematical notations,
our procedural thoughts will be expressed in Lisp.
Lisp was invented in the late 1950s as a formalism for reasoning about the use of certain kinds of logical expressions,
called recursion equations, as a model for computation.
The language was conceived by John McCarthy and is based on his paper
\quotation{Recursive Functions of Symbolic Expressions and Their Computation by Machine} (McCarthy 1960).

Despite its inception as a mathematical formalism, Lisp is a practical programming language.
A Lisp interpreter is a machine that carries out processes described in the Lisp language.
The first Lisp interpreter was implemented by
McCarthy with the help of colleagues
and students in the Artificial Intelligence Group of the MIT Research Laboratory of Electronics
and in the MIT Computation Center.
\footnote{%
   The Lisp 1 Programmer's Manual appeared in 1960,
   and the Lisp 1.5 Programmer's Manual (McCarthy 1965) was published in 1962.
   The early history of Lisp is described in McCarthy 1978.
}
Lisp, whose name is an acronym for LISt Processing,
was designed to provide symbol-manipulating capabilities for attacking programming problems
such as the symbolic differentiation and integration of algebraic expressions.
It included for this purpose new data objects known as atoms and lists,
which most strikingly set it apart from all other languages of the period.

Lisp was not the product of a concerted design effort.
Instead, it evolved informally in an experimental manner
in response to users' needs and to pragmatic implementation considerations.
Lisp's informal evolution has continued through the years,
and the community of Lisp users has traditionally resisted attempts
 to promulgate any \quotation{official} definition of the language.
This evolution, together with the flexibility and elegance of the initial conception,
has enabled Lisp, which is the second oldest language in widespread use today (only Fortran is older),
to continually adapt to encompass the most modern ideas about program design.
Thus, Lisp is by now a family of dialects, which,
while sharing most of the original features,
may differ from one another in significant ways.
The dialect of Lisp used in this book is called Scheme.
\footnote{%
   The two dialects in which most major Lisp programs of the 1970s were written are
   MacLisp (Moon 1978; Pitman 1983),developed at the MIT Project MAC, and Interlisp (Teitelman 1974),
   developed at Bolt Beranek and Newman Inc.
   and the Xerox Palo Alto Research Center.
   Portable Standard Lisp (Hearn 1969; Griss 1981)
   was a Lisp dialect designed to be easily portable between different machines.
   MacLisp spawned a number of subdialects,
   such as Franz Lisp,
   which was developed at the University of California at Berkeley,
   and Zetalisp (Moon 1981),
   which was based on a special-purpose processor designed
   at the MIT Artificial Intelligence Laboratory to run Lisp very efficiently.
   The Lisp dialect used in this book, called Scheme (Steele 1975),
   was invented in 1975 by Guy Lewis Steele Jr. and Gerald Jay Sussman of the MIT Artificial Intelligence Laboratory
   and later reimplemented for instructional use at MIT.
   Scheme became an IEEE standard in 1990 (IEEE 1990).
   The Common Lisp dialect (Steele 1982, Steele 1990) was developed by the Lisp community to combine features
   from the earlier Lisp dialects to make an industrial standard for Lisp.
   Common Lisp became an ANSI standard in 1994 (ANSI 1994).
}

Because of its experimental character and its emphasis on symbol manipulation,
Lisp was at first very inefficient for numerical computations,
at least in comparison with Fortran.
Over the years, however,
Lisp compilers have been developed that translate programs into machine code
that can perform numerical computations reasonably efficiently.
And for special applications, Lisp has been used with great effectiveness.
\footnote{%
   One such special application was a breakthrough computation of scientific importance
   -- an integration of the motion of the Solar System that extended previous results by nearly two orders of magnitude,
   and demonstrated that the dynamics of the Solar System is chaotic.
   This computation was made possible by new integration algorithms, a special-purpose compiler,
   and a special-purpose computer all implemented with the aid of software tools written in Lisp
   (Abelson et al. 1992; Sussman and Wisdom 1992).
}
Although Lisp has not yet overcome its old reputation as hopelessly inefficient,
Lisp is now used in many applications where efficiency is not the central concern.
For example, Lisp has become a language of choice for operating-system shell languages
and for extension languages for editors and computer-aided design systems.

If Lisp is not a mainstream language,
why are we using it as the framework for our discussion of programming?
Because the language possesses unique features
that make it an excellent medium for studying important programming constructs
and data structures and for relating them to the linguistic features that support them.
The most significant of these features is the fact that Lisp descriptions of processes,
called procedures,
can themselves be represented and manipulated as Lisp data.
The importance of this is that there are powerful program-design techniques
that rely on the ability to blur the traditional distinction
between \quotation{passive} data and \quotation{active} processes.
As we shall discover,
Lisp's flexibility in handling procedures as data
makes it one of the most convenient languages in existence for exploring these techniques.
The ability to represent procedures as data also makes Lisp an excellent language
for writing programs that must manipulate other programs as data,
such as the interpreters and compilers that support computer languages.
Above and beyond these considerations, programming in Lisp is great fun.
