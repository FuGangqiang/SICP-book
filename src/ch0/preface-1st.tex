% ch0-preface-1st

\chapter[chap:preface1]{Preface to the First Edition}

A computer is like a violin.
You can imagine a novice trying first a phonograph and then a violin.
The latter, he says, sounds terrible.
That is the argument we have heard from our humanists and most of our computer scientists.
Computer programs are good, they say, for particular purposes, but they aren't flexible.
Neither is a violin, or a typewriter, until you learn how to use it.

Marvin Minsky,
\quotation{Why Programming Is a Good \par Medium for Expressing Poorly-Understood and Sloppily-Formulated Ideas}

\quotation{The Structure and Interpretation of Computer Programs}
is the entry-level subject in computer science at the Massachusetts Institute of Technology.
It is required of all students at MIT who major in electrical engineering or in computer science,
as one-fourth of the \quotation{common core curriculum,}
which also includes two subjects on circuits and linear systems and a subject on the design of digital systems.
We have been involved in the development of this subject since 1978,
and we have taught this material in its present form since the fall of 1980 to between 600 and 700 students each year.
Most of these students have had little or no prior formal training in computation,
although many have played with computers a bit and a few have had extensive programming or hardware-design experience.

Our design of this introductory computer-science subject reflects two major concerns.
First,
we want to establish the idea that
a computer language is not just a way of getting a computer to perform operations
but rather that it is a novel formal medium for expressing ideas about methodology.
Thus, programs must be written for people to read, and only incidentally for machines to execute.
Second,
we believe that the essential material to be addressed by a subject at this level
is not the syntax of particular programming-language constructs,
nor clever algorithms for computing particular functions efficiently,
nor even the mathematical analysis of algorithms and the foundations of computing,
but rather the techniques used to control the intellectual complexity of large software systems.

Our goal is that students who complete this subject
should have a good feel for the elements of style and the aesthetics of programming.
They should have command of the major techniques for controlling complexity in a large system.
They should be capable of reading a 50-page-long program, if it is written in an exemplary style.
They should know what not to read, and what they need not understand at any moment.
They should feel secure about modifying a program, retaining the spirit and style of the original author.

These skills are by no means unique to computer programming.
The techniques we teach and draw upon are common to all of engineering design.
We control complexity by building abstractions that hide details when appropriate.
We control complexity by establishing conventional interfaces that
enable us to construct systems by combining standard, well-understood pieces in a \quotation{mix and match} way.
We control complexity by establishing new languages for describing a design,
each of which emphasizes particular aspects of the design and deemphasizes others.

Underlying our approach to this subject is our conviction that
\quotation{computer science} is not a science and that its significance has little to do with computers.
The computer revolution is a revolution in the way we think and in the way we express what we think.
The essence of this change is the emergence of what might best be called procedural epistemology
-- the study of the structure of knowledge from an imperative point of view,
as opposed to the more declarative point of view taken by classical mathematical subjects.
Mathematics provides a framework for dealing precisely with notions of \quotation{what is.}
Computation provides a framework for dealing precisely with notions of \quotation{how to.}

In teaching our material we use a dialect of the programming language Lisp.
We never formally teach the language, because we don't have to.
We just use it, and students pick it up in a few days.
This is one great advantage of Lisp-like languages:
They have very few ways of forming compound expressions, and almost no syntactic structure.
All of the formal properties can be covered in an hour, like the rules of chess.
After a short time we forget about syntactic details of the language (because there are none)
and get on with the real issues -- figuring out
what we want to compute,
how we will decompose problems into manageable parts,
and how we will work on the parts.
Another advantage of Lisp is that
it supports (but does not enforce) more of the large-scale strategies for modular decomposition of programs
than any other language we know.
We can make procedural and data abstractions,
we can use higher-order functions to capture common patterns of usage,
we can model local state using assignment and data mutation,
we can link parts of a program with streams and delayed evaluation,
and we can easily implement embedded languages.
All of this is embedded in an interactive environment
with excellent support for incremental program design, construction, testing, and debugging.
We thank all the generations of Lisp wizards,
starting with John McCarthy, who have fashioned a fine tool of unprecedented power and elegance.

Scheme, the dialect of Lisp that we use,
is an attempt to bring together the power and elegance of Lisp and Algol.
From Lisp we take the metalinguistic power that derives from
the simple syntax,
the uniform representation of programs as data objects,
and the garbage-collected heap-allocated data.
From Algol we take lexical scoping and block structure,
which are gifts from the pioneers of programming-language design who were on the Algol committee.
We wish to cite John Reynolds and Peter Landin for their insights into
the relationship of Church's lambda calculus to the structure of programming languages.
We also recognize our debt to the mathematicians
who scouted out this territory decades before computers appeared on the scene.
These pioneers include Alonzo Church, Barkley Rosser, Stephen Kleene, and Haskell Curry.
